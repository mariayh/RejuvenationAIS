% !TEX TS-program = xelatex
% !TEX encoding = UTF-8
%%
%% This is file `thesis.tex',
%% generated with the docstrip utility.
%%
%% The original source files were:
%%
%% nudtpaper.dtx  (with options: `thesis')
%% 
%% This is a generated file.
%% 
%% Copyright (C) 2010 by Liu Benyuan <liubenyuan@gmail.com>
%% 
%% This file may be distributed and/or modified under the
%% conditions of the LaTeX Project Public License, either version 1.3a
%% of this license or (at your option) any later version.
%% The latest version of this license is in:
%% 
%% http://www.latex-project.org/lppl.txt
%% 
%% and version 1.3a or later is part of all distributions of LaTeX
%% version 2004/10/01 or later.
%% 
%% To produce the documentation run the original source files ending with `.dtx'
%% through LaTeX.
%% 
%% Any Suggestions : LiuBenYuan <liubenyuan@gmail.com>
%% Thanks Xue Ruini <xueruini@gmail.com> for the thuthesis class!
%% Thanks sofoot for the original NUDT paper class!
%% 
%1. 规范硕士导言
% \documentclass[master,ttf]{nudtpaper}
%2. 规范博士导言
% \documentclass[doctor,twoside,ttf]{nudtpaper}
%3. 如果使用是Vista
% \documentclass[master,ttf,vista]{nudtpaper}
%4. 建议使用OTF字体获得较好的页面显示效果
%   OTF字体从网上获得,各个系统名称统一,不用加vista选项
%   如果你下载的是最新的(1201)OTF英文字体,建议修改nudtpaper.cls,使用PS Std
% \documentclass[doctor,twoside,otf]{nudtpaper}
%5. 如果想生成盲评,传递anon即可,仍需修改个人成果部分
% \documentclass[master,otf,anon]{nudtpaper}
%
\documentclass[doctor,twoside,otf]{whupaper}
\usepackage{mynudt}

\classification{TP18} %分类号
\confidentiality{公开} %密级
\UDC{004.8}       %UDC
\serialno{10486} %编号:10486是武汉大学的高校编号
\title{钢铁生产调度中的免疫优化算法研究}
\displaytitle{钢铁生产调度中的免疫优化算法研究} %页面中显示的内容
\author{吕\quad{}林}
\zhdate{\zhtoday}
\entitle{Research on  Immune Optimization Algorithms  under The Scheduling of Steel Production}
\enauthor{Lu Lin}
\endate{\entoday}
\subject{计算机应用技术} %专业
\ensubject{Computer Application Technology}
\researchfield{人工免疫}%研究方向
\supervisor{梁意文\quad{}教授}%指导教师姓名、职称
%\cosupervisor{王五\quad{}副教授} % 没有就空着
\ensupervisor{Professor Liang Yiwen}
%\encosupervisor{}
%\papertype{工学}
\papertype{}
%\enpapertype{Engineering}

\begin{document}
\graphicspath{{figures/}}
% 制作封面,生成目录,插入摘要
\maketitle
\frontmatter
\begin{creation}
\begin{enumerate}[{}1{)}]
\item 创新点1
\item 创新点2。
\item 创新点3
\item 创新点4
\end{enumerate}
\end{creation}


 %放在frontmatter前面会有章节号,但是怎样让它不在目录里面出现呢?

\midmatter
\begin{cabstract}
中文摘要

\end{cabstract}
\ckeywords{人工免疫; 优化算法; 钢铁生产}

\begin{eabstract}
Here is the English abstract.
\end{eabstract}
\ekeywords{Artificial Immune; Optimization Algorithms; Steel-Making}


\tableofcontents
\listoftables
\listoffigures
%\begin{denotation}

\item[HPC] 高性能计算 (High Performance Computing)
\item[cluster] 集群
\item[Itanium] 安腾
\item[SMP] 对称多处理
\item[API] 应用程序编程接口
\item[PI]	聚酰亚胺
\item[MPI]	聚酰亚胺模型化合物,N-苯基邻苯酰亚胺
\item[PBI]	聚苯并咪唑
\item[MPBI]	聚苯并咪唑模型化合物,N-苯基苯并咪唑
\item[PY]	聚吡咙
\item[PMDA-BDA]	均苯四酸二酐与联苯四胺合成的聚吡咙薄膜
\item[$\Delta G$]  	活化自由能~(Activation Free Energy)
\item [$\chi$] 传输系数~(Transmission Coefficient)
\item[$E$] 能量
\item[$m$] 质量
\item[$c$] 光速
\item[$P$] 概率
\item[$T$] 时间
\item[$v$] 速度

\end{denotation}
 %符号列表

%书写正文,可以根据需要增添章节; 正文还包括致谢,参考文献与成果
\mainmatter
\chapter{绪论}



\section{研究背景和意义}
正文内容

\subsection{subsection}


\subsubsection{(1.1.1.1 题目)}
正文内容

正文内容

正文内容

\subsubsection{(1.1.1.2 题目)}
正文内容

正文内容

正文内容

\subsection{(1.1.2 题目)}
正文内容

正文内容



\section{研究动机}
正文内容



\section{主要工作和创新点}
正文内容



\subsection{(1.3.1 题目)}
正文内容


\section{论文的组织结构}

\chapter{钢铁生产调度研究现状}

\section{字体段落}

\section{表格明细}

%\input{data/chap03}
%\input{data/chap04}
%\input{data/chap05}
%\input{data/chap06}

\cleardoublepage
\phantomsection
\addcontentsline{toc}{chapter}{参考文献}
\bibliographystyle{bstutf8}
\bibliography{ref/refs}

\begin{resume}

  \section*{发表的学术论文} % 发表的和录用的合在一起

  \begin{enumerate}[{[}1{]}]
  \addtolength{\itemsep}{-.36\baselineskip}%缩小条目之间的间距,下面类似
  \item Lin Lu, Yiwen Liang, Chao Yang, Shiwei Song. Detecting Faults in IP-PBX with Deterministic Dendritic Cell Algorithm. AISS: Advances in Information Sciences and Service Sciences, Vol. 3, No. 11, pp. 457 ~ 465, 2011. (EI 源刊, 检索号:20120114659553.)
  \item Lin Lu,Hongbin Dong,Chao Yang, Daihua Yangl. A Novel  Mass Data Processing Framework based on Hadoop for Electrical Power Monitoring System. 2012 Asia-Pacific Power and Energy Engineering Conference. (Accepted.)
  \item Lin Lu, Yiwen Liang, He Yang, Chao Yang. Danger Theory: A New Approach in Big Data Analysis. The International Conference on Automatic Control and Artificial Intelligence (ACAI2012) (Accepted.) 
  \end{enumerate}

%  \section*{研究成果} % 有就写,没有就删除
%  \begin{enumerate}[{[}1{]}]
%  \addtolength{\itemsep}{-.36\baselineskip}%
%  \item 任天令, 杨轶, 朱一平, 等. 硅基铁电微声学传感器畴极化区域控制和电极连接的
%    方法: 中国, CN1602118A. (中国专利公开号.)
%  \item Ren T L, Yang Y, Zhu Y P, et al. Piezoelectric micro acoustic sensor
%    based on ferroelectric materials: USA, No.11/215, 102. (美国发明专利申请号.)
%  \end{enumerate}
\end{resume}
 %博士期间取得的学术成果
% 最后,需要的话还要生成附录,全文随之结束。
\appendix
\backmatter
%%% Local Variables: 
%%% mode: latex
%%% TeX-master: "../main"
%%% End: 

\chapter{术语表}
\label{cha:engorg}
%\section[First Principles]{first principles}

%\subsection{Typography exists to honor content.}
%
%
%
%\subsection{Letters have a life and dignity of their own.}



%\chapter{其它附录}
%前面两个附录主要是给本科生做例子。其它附录的内容可以放到这里,当然如果你愿意,可
%以把这部分也放到独立的文件中,然后将其 \verb|\input| 到主文件中。


%%% Local Variables:
%%% mode: latex
%%% TeX-master: "../main"
%%% End:

\begin{ack}
  

\end{ack}


\end{document}
